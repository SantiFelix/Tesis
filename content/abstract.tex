% !TEX root = ../thesis-example.tex
%

\pdfbookmark[0]{Abstract}{Abstract}
\chapter*{Resumen}
\label{sec:abstract}
%\vspace*{-10mm}

%\blindtext

El presente trabajo de tesis presenta la identificacion de distribuciones estadisticas en datos de mercados financieros. 

En el presente trabajo de tesis se presentan y discuten los resultados obtenidos tras aplicar un filtro de entropía a series de tiempo de mercados financieros.\\

 
%\vspace*{20mm}
Las técnicas de procesamiento de datos empleadas son básicamente dos, medias móviles y entropía. Las medias móviles son empleadas para poder interpretar de una mejor manera los datos tras aplicar un filtro de entropía ya que permiten ajustar una ventana temporal de los mercados. Los indicadores de precios al mercado utilizados fueron Dow Jones, DAX Performance, BMV IPC, y Nikkei 225.\\
%{\usekomafont{chapter}Abstract (en ingles)}\label{sec:abstract-diff} \\

Las medias móviles son mejor conocidas por los analistas de datos como Moving Average o MAV, la cual obtiene un promedio de un determinado conjunto de datos y esto a su vez permite aminorar el ruido de la serie de tiempo, para que al aplicar el filtro de entropía el resultado obtenido no manifieste un mayor número de máximos o un comportamiento mayormente uniforme.\\

La selección de las ventanas de tiempo se hace manualmente por el autor, las cuales se pueden realizar para cualquier ventana de tiempo bajo dos régimenes, el filtrado de entropía en los precios del mercado a través de un Moving Average, o el filtrado de entropía en los precios del mercado sin un Moving Average.\\
%\blindtext

Se utilizó software que permitiera hacer los cálculos y un análisis más preciso en menor tiempo, esto permitió identificar que es posible ajustar distribuciones estadísticas a los resultados obtenidos para la distribuciones después de haber aplicado un filtro de entropía con o sin medias móviles. %\textregistered 
%\clearpage
\cleardoublepage