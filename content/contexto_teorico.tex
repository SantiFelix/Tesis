% !TEX root = ../thesis-example.tex
%
\chapter{Contexto teorico}
\label{contexto}

\cleanchapterquote{Un sistema físico macroscópico no puede ser descrito solo en términos de variables mecánicas, la llamada entropía era necesaria}{P. Richmond, J. Mimkes, S. Hutzler}{(2013)}

%\Blindtext[1][1]
%\begin{figure}[htp!]
%\centering
%\includegraphics[scale=0.5]{gfx/chikorita.png}
%\caption{Aqui esta chikorita}
%\end{figure}

\section{La similitud entre economia y fisica}

\subsection{Analogías Física-Economía} 

En \textit{Física}, un sistema cerrado conserva la energía,  en \textit{Economía}, un sistema económico cerrado conserva el \textit{dinero}. \newline

\textit{De la anterior analogía se entiende que no hay un flujo externo de dinero, por lo que la cantidad total de dinero se conserva}.

La \textit{mecánica estadísitica} estudiada por los físicos, y la economía, tienen en común que ambas estudian grandes ensambles; colecciones de átomos, y agentes económicos, respectivamente.
(Cottrell, Michelson, Wright, Classical Econophysics, pg.149)

\textit{De la anterior analogía, en una colección de átomos se asume que cada uno de ellos posee una energía cinética $\mathit{\varepsilon}_{i} \geqslant 0 $, mientras que en los mercados financieros cada uno de los agentes participantes tiene un recurso llamado dinero el cual no puede ser negativo, es decir  $\mathit{d}_{i} \geqslant 0 $}.(Cottrell, Michelson, Wright, Classical Econophysics, pg.149)

Vale la pena mencionar que un ensamble microcanónico es un conjunto  en el que las características y el comportamiento de los componentes son similares, posee entropía máxima y conservación de energía. Gracias a su naturaleza física,es posible estudiarle como un todo.  


Adicionalmente, en \textit{Física}, a menudo se dice que un sistema se encuentra en equilibrio cuando la energía con que interactúan los componentes es conservada. 


\subsection{Economía neoclásica} 

En los comienzos de la \textit{Econofísica} se encuentran emblemáticos personajes como Adolphe Quétlet(1796-1874), Léon Walras (1834-1910), Vilfredo Pareto (1848-1923), y Adam Smith (1723-1790), quienes ofrecieron un conjunto de ideas que permitieron ampliar la manera en que se estudia la \textit{Economía}, y que más allá de considerar a una \textit{transacción} como una consecuencia de carácter determinista debida a la interacción de intercambio de bienes entre dos agentes, considera la toma de decisiones entre los agentes en función de aspectos sociales, factores externos a los mercados, e incluso de la distribucipon de la riqueza.
\newline

Esta brillante abertura permite a la \textit{Econofísica} estudiar los sistemas complejos intrínsecos de la \textit{Economía} a partir de que a cada sistema complejo en que se encuentra inmersa le componen un gran número de \textit{grados de libertad} en diferentes escalas de tiempo(Richmond Econophysics pg 17).
\newline

Asi como Pareto estudió la distribución de la riqueza en el siglo XIX, Adam Smith propuso la descripción de un fenómeno constitutivo en este trabajo de tesis, llamado \textit{la mano invisible}. Este y los siguientes conceptos son parte de la \textit{Economía neoclásica}, su naturaleza conlleva el pensamiento de que la \textit{Economía neoclásica} es quien da origen a la \textit{Econofísica}.

%---------------------------------------------------------------------------
%Introducir RETORNOS en el texto
%Se define como  r($\mathit{t_{i}}$) $\equiv $ \textit{x}$(\mathit{t}_{i})$ - \textit{x}($\mathit{t}_{i} - \Delta t$) , donde \textit{x}($\mathit{t}_{i}$ es una secuencia de precios logarítmicos 
Los retornos son usualmente variables de análisis más rentables que el precio perse, por varias razones. La distribución de los retornos es más simétrica y estable en el tiempo que la distribución de los precios en sí. La estructura de los retornos es cercana a estacionaria, mientras que la estructura de los precios observados en el tiempo no lo es (Dacoragna, Et Al, 2001).
\newpage

El logaritmo de los precios muestra que están normalmente distribuídos, al menos aproximadamente (M.F.M. Osborne, 1959). En este trabajo de tesis se trabaja con los retornos de los registros diarios individuales de cada serie de tiempo financiera, en otras palabras, la ventana temporal entre cada \guillemotleft tick \guillemotright es constante y se le conoce como \textit{lag} la cual se denota como $\delta t$. 
\newline

Los retornos son expresados como:
\begin{center}
$R_t=P_{t+1}-P_t$
\end{center}
Es decir, el retorno del precio en cuestión ( $R_t$ ) es equivalente al precio individual de su sucesor temporal ( $P_{t+1}$ ) menos el precio del objeto temporal en turno ( $P_t$ ), sin embargo, siguiendo los ideales de Osborne, hay que calcular el logaritmo de los retornos a sabiendas de que los precios se etiquetan como $\mathit{Z_t}$ y son las variables de registro de las mediciones individuales diarias al cierre del mercado, donde \newline $t \in \mathcal{T} = \{\mathit{T_1,T_2,\ldots , T_M}\}$. El logaritmo del precio es:

\begin{center}
$\mathit{Y_{t}} = $ ln $\mathit{Z_{t}}$
\end{center}

Entonces los \textit{retornos logarítmicos} son definidos como: 

\begin{center}
$X_t = $ $\mathit{Y_{t}} - $  $\mathit{Y_{t - 1}}$
\end{center}

Cuya interpretación es la diferencia sucesiva de los logaritmos de los precios  registrados en el tiempo, en otras palabras, la diferencia del logaritmo del precio en turno y su antecesor temporal. Aunque en términos computacionales es mucho más sencillo trabajar con la siguiente expresión que respeta la proposición de Osborne (Frantisek Slanina, 2014): 
\begin{center}
ln $R_t = $ ln $\mathit{P_{t+1}} - $ ln $\mathit{P_{t}}$
\end{center}%esto fue sacado de la pagina 132 y 133 del libro essentials of econophysics modeling


\subsection{Ley de Boltzmann-Gibbs} 

La \textit{ley de la distribución de la probabilidad de la energía} en la \textit{mecánica estadística}:

\begin{center}
$\mathit{P(\varepsilon)} = $ C $e^{-\varepsilon/\mathit{T}}$ 
\end{center}
Donde:

$\varepsilon$ es la energía

$C$ es una constante de normalización

$\mathit{T}$ es la temperatura (Wannier,1966) (Cottrell, Michelson, Wright, Classical Econophysics, pg.149)

Dada la similitud de un sistema descrito por la Ley de Boltzmann-Gibbs y las condiciones de un sistema económico cerrado, se puede esta ley a una descripción de la \textit{distribución de la probabilidad del dinero}
(Cottrell, Michelson, Wright, Classical Econophysics, pg.149).


\subsection{1.16.3. Distribución de la probabilidad del dinero (en equilibrio)} 


De acuerdo con Cottrel, Michelson y Wright (Cottrell, Michelson, Wright, Classical Econophysics, pg.149) la distribución de la probabilidad del dinero sigue la forma de la \textit{Ley de Boltzmann-Gibbs}


\begin{center}
$\mathit{P(d)} = $ C $e^{-d/\mathit{T}}$ 
\end{center}
Donde:

$d$ es el dinero

$C$ es una constante de normalización

$\mathit{T}$ es el promedio de la cantidad de dinero por agente en el sistema económico.

Con base en el artículo \textit{Price variations in a stock market with many agents} por Shubik, Pakzuski y Bak en 1997, la ley de la conservación del dinero establece que el dinero no puede ser manufacturado por agentes del sistema económico pero si puede ser transferido entre ellos. Esta descripción es análoga a la que un sistema conservativo presenta en Física, por ejemplo, cuando en un sistema cerrado sin intercambio de energía con el exterior, los átomos colisionan entre sí. 
\newpage

\subsection{Ley de la conservación del dinero.} 

En relación con lo establecido por Shubik, Pakzuski y Bak en 1997 en el artículo ya mencionado, lo que compete en este trabajo de tesis es el resultado de la interacción entre los agentes \textit{i} y \textit{j}. 

Sean \textit{i} y \textit{j} dos agentes que interactúan en el mercado, correspondiéndole a cada uno de ellos una cantidad finita de dinero $d_{i}$ y $d_{j}$ respectivamente, se puede denotar de la siguiente manera: $[d_{i},d_{j}]$. 

Si al llevarse a cabo la interacción entre dos agentes y el intercambio de dinero que llevan a cabo es constante, dicho intercambio se etiqueta como $\Delta d$. El resultado de la interacción de los agentes en cuestión se expresa como:  $[d_{i},d_{j}]$ $\longrightarrow$  $[d'_{i},d'_{j}] = [d_{i} - \Delta d ,d_{j} + \Delta d]$. Dicho lo anterior puede notarse que $d_{i} + d_{j} = d'_{i} + d'_{j}$ lo cual significa que la cantidad total de dinero en la transacción es conservada si no existe un flujo externo de dinero que modifique la cantidad total $d$.

En tales condiciones se asume que la distribución de probabilidad del dinero en equilibrio es invariante pese a fuertes fluctuaciones $\Delta d$ entre los agentes. (Cottrell, Michelson, Wright, Classical Econophysics, pg.149)


%Ahora que se ha comprendido que el intercambio de dinero $\Delta d$ entre %los agentes es la interacción elemental para la existencia de la %\textit{Economía}, es momento de entender el funcionamiento de la %\textit{Economía neoclásica} para poder utilizar el concepto de %\textit{caminata aleatoria} mismo que se utiliza para sentar las bases del %concepto de \textit{mercado eficiente}.

\subsection{Segunda Ley de la termodinámica.} 



