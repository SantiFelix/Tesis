% !TEX root = ../thesis-example.tex
%
\chapter{Introducción}
\label{sec:intro}
Se puede asumir que para tener un mayor desarrollo humano hay que poseer avances en la ciencia, con base en un entendimiento de las limitaciones que existen hoy en día, para así poder revolucionar la tecnolog\'ia, de manera que al tener un nuevo avance tecnológico se dé pie a nuevos avances científicos, grosso modo es así como se genera un círculo virtuoso, no obstante el trasfondo es más meritorio que codicioso.
\newline

Si bien la economía mundial se mueve a través de intereses particulares de las diferentes naciones, empresas, casas de bolsa, mercados o incluso de los intereses propios de cada individuo, la Física no es precisamente motivada por los mismos intereses, ya que al tratarse de una ciencia natural, las leyes de la naturaleza y en particular las leyes físicas que describen la naturaleza del universo son asunto interés común, y es ahí donde al unir conocimientos de propiedad universal a la economía puede parecer incluso utópico. \newline

Es bien sabido que el estudio de la naturaleza y todo lo que le conforma ha partido de los principios básicos del razonamiento humano, hasta llegar a comprender la din\'amica de los objetos que hay alrededor de cada individuo, conforme se concibe dicha descripción f\'isica de un fen\'omeno se puede profundizar m\'as en el comportamiento del fenómeno, gracias a un entero entendimiento del movimiento de los cuerpos en el universo se ha podido comprender la naturaleza del movimiento en cuerpos de mayor tamaño, como lo son planetas, tambi\'en se ha podido ir un paso m\'as all\'a y comenzar a interpretar fen\'omenos tales como el modelado de un sistema de part\'iculas con carga, e incluso adentrarse en el mundo microsc\'opico y modelar fenómenos a escalas que resultaban dif\'iciles de imaginar hace un siglo, y lo que es más meritorio es que hoy en día es posibile estudiar grandes cantidades de información y comprender su comportamiento.
\newline


En diferentes \'ambitos, desde los microsc\'opico-macrosc\'opico, hasta la complejidad en el estudio de los modelos de agentes, la F\'isica se ha encargado de explicar los fen\'omenos que competen a la naturaleza del universo, la Econom\'ia no es la excepci\'on, puesto que es un fen\'omeno m\'as en el que la humanidad se ve envuelta d\'ia a d\'ia, tal como lo es la termodin\'amica del medio en que los sere humanos habitamos el planeta, la econom\'ia es tambi\'en algo de lo que nadie est\'a exento de su sustancial car\'acter en el desarrollo de nuestra especie.
\newline

Desde la teor\'ia econ\'omica de Adam Smith en 1767 (Nordhaus, Samuelson, 2004), hecho que revolucion\'o a la econom\'ia, y es por ello que se le conoce como el padre de la econom\'ia moderna, puesto que al d\'ia de hoy mucho de su legado se sigue aplicando de una u otra manera, evidentemente con variantes pero existe, hasta el más reciente concepto de entropía (entropía de la información o de Shannon) se encuentra particular interés en la unión de dos ciencias que describen el funcionamiento de los sistemas en que nuestra especie se desenvuelve, ya que con las herramientas de una ciencia se puede explicar cualitativa y cuantitavamente un fenómeno que pareciera ser exclusivo de una sola ciencia.
\newline

Aunque no es estrictamente necesario entender los conceptos del \'area de econom\'ia, es fundamental entender que los mercados son sistemas formados por un n\'umero muy grande de componentes, o participantes, cada uno viendo por su propio inter\'es, sin saber que al hacerlo est\'an promoviendo un concepto conocido como \textit{la mano invisible}. Al tratarse de un fen\'omeno donde se experimenta alta competitividad por parte de los otros internautas, y a\'un conociendo el funcionamiento de los indicadores de los mercados, o estando informado sobre los acontecimientos que influyen en las decisiones de los "traders", existe una gran incertidumbre sobre si es posible predecir una tendencia que sea favorable, esto debido al gran volumen de información que muestra una serie de tiempo financiera. 
\newline

Con esto en mente esta tesis esta organizada de la siguiente manera:\newline 
En el capítulo 1 se introduce al tema de la econofísica donde se tratarán los fundamentos que conforman a esta nueva rama de estudio, y se mencionan conceptos básicos de economía.
\newline

En el capítulo 2 se presentan las características de los mercados estudiados, y se describe como funcionan los mercados con base en la perspectiva de la econofísica, planteando el objetivo general de este trabajo, así como los objetivos particulares.\newline

En el capítulo \ref{Metodologia} se explica el método empleado para la obtención de entropía de los mercados, se describe brevemente la manera de obtener la mínima entropía de las series de tiempo financieras.
\newline

Y finalmente en el capítulo 4 se presentan los resultados y conclusiones de este trabajo.


%\cleanchapterquote{You can’t do better design with a computer, but you can speed up your work enormously.}{Wim Crouwel}{(Graphic designer and typographer)}

%\Blindtext[2][2]

%\section{Postcards: My Address}
%\label{sec:intro:address}

%\textbf{Ricardo Langner} \\
%Alfred-Schrapel-Str. 7 \\
%01307 Dresden \\
%Germany


%\section{Motivation and Problem Statement}
%\label{sec:intro:motivation}

%\Blindtext[3][1] \cite{Jurgens:2000,Jurgens:1995,Miede:2011,Kohm:2011,Apple:keynote:2010,Apple:numbers:2010,Apple:pages:2010}

%\section{Results}
%\label{sec:intro:results}

%\Blindtext[1][2]

%\subsection{Some References}
%\label{sec:intro:results:refs}
%\cite{WEB:GNU:GPL:2010,WEB:Miede:2011}
%
%\section{Thesis Structure}
%\label{sec:intro:structure}
%
%\textbf{Chapter \ref{sec:related}} \\[0.2em]
%\blindtext
%
%\textbf{Chapter \ref{sec:system}} \\[0.2em]
%\blindtext
%
%\textbf{Chapter \ref{sec:concepts}} \\[0.2em]
%\blindtext
%
%\textbf{Chapter \ref{sec:concepts}} \\[0.2em]
%\blindtext
%
%\textbf{Chapter \ref{sec:conclusion}} \\[0.2em]
%\blindtext
