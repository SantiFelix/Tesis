% !TEX root = ../thesis-example.tex
%
\chapter{Conceptos}
\label{sec:related}

\cleanchapterquote{Un sistema físico macroscópico no puede ser descrito solo en términos de variables mecánicas, la llamada entropía era necesaria}{P. Richmond, J. Mimkes, S. Hutzler}{(2013)}




{
    \noindent
 Los conceptos que se presentan a continuación son manifestados en el orden en que se tenga el mayor entendimiento de este trabajo de Tesis.
}

{
\noindent
\subsection{Econofísica}
}

Término acuñado por Eugene Stanley en 1997 para referirse a una rama de la Física\citep[][página 3]{roehner2002patterns} que a partir de encontrar relaciones entre leyes físicas y variables macroscópicas propias de la Economía, busca un entendimiento más profundo de los fenómenos que se presentan en los mercados. 


{
\noindent
\subsection{2.1.1. Economía} 
}

Es el estudio de cómo las sociedades utilizan recursos escasos para producir bienes valiosos y distribuirlos entre diferentes personas \citep[][]{samuelson2009economics}.


{
\noindent
\subsection{Agentes} 
}

Son los individuos, compañías, países, casas de bolsa, familias, comercientes, inversionistas, capaces de interactuar entre sí de una manera no trivial, que además se encuentran inmersos en un mecanismo auto-organizado pese a poseer intereses particulares cada uno.
\newpage

{
\noindent
\subsection{Modelo de agentes} 
}

Estructuras presentes en cualquier ciencia, auqneue stas estructuras tienen fenómenos diametralmente diferentes según sea el campo de estudio, poseen datos que pueden  o no exhibir características similares tales como; un número N de agentes, la estadística no Gaussiana en cortos periodos de tiempo, adaptabilidad, autocorrelaciones, autointeracción, y la inaccesibilidad al análisis de un solo agente en particular \citep[][]{Boucha}.


{
\noindent
\subsection{Modelo de agentes en econofísica} 
}

De acuerdo con Bouchaud y Cont \citep[][]{Boucha} es un modelos donde el intercambio es caracterizado simplemente por dos tipos de agentes, traders técnicos y fundamentales. Toda la demanda para un mercado es alcanzada por estos dos grupos de agentes. Con dicha caracterización los autores esperaban recopilar el comportamiento básico individual y colectivo de los agentes.  


{
\noindent
\subsection{ Sistema complejo} 
}

Compuesto por un número muy grande de agentes, los elementos que le conforman poseen dinámica en sus interacciones entre sí, esto con base en los grados de libertad que el modelo en cuestión les permita, de tal manera que sus interacciones no son triviales. También contiene regularidades como lo es la auto-organización. Ver \citep[][página 8]{richmond}


{
\noindent
\subsection{ Precio} 
}

Cuota requerida para el proceso de adquisición, o para que se lleve a cabo un trato de concesión de un bien. Dependiendo del contexto en el modelo de agentes y del sistema complejo que se trate, puede ser una variable dinámica en la que haya intermediarios que le modifican directa e indirectamente. 


{
\noindent
\subsection{1.5. Precio en econofísica} 
}

En este contexto encontramos que es la unidad en que se \guillemotleft  mide\guillemotright una variable dinámica conformada por N agentes de un respectivo modelo, mismos que disponen de una gama de posibilidades con respecto a los grados de libertad de dicho entorno en que se lleven a cabo los tratos de adquisición, o concesión de un bien (interacciones). Los N agentes causantes de dicha variabilidad pueden ser pertencientes o no a dicho modelo, estas interacciones entre agentes internos o externos tienen una retroalimentación cuyo efecto es la principal causa de que su comportamiento sea el de un sistema complejo.

\newpage

{
\noindent
\subsection{2.3.  Dinero} 
}

Cada agente \textit{i} posee una simbólica suma de recurso que utiliza para adquirir bienes o comodidades de consumo, su unidad de medida es la \guillemotleft moneda \guillemotright aunque esta es una unidad arbitrariamente divisible. La moneda perse no puede ser producida ni consumida por los actores   (Cockshitt W. P, Cottrell A. F., Michaelson G. J., Et  Al, 2009). Los agente intercambian \textit{dinero}  y obtienen un incremento cuando llevan a cabo tratos de venta o cesión, y obtienen un detrimento cuando llevan a cabo un trato de adquisición o compra de un bien. 



{
\noindent
\subsection{2.4. Economía financiera} 
}

Rama de la economía que analiza cómo los inversionistas racionales deben invertir sus bienes para alcanzar sus objetivos de la mejor manera posible  (Samuelson \& Nordhaus, 2005).


{
\noindent
\subsection{2.5. Hechos estilizados} 
}

Aunque no se profundiza en este concepto, vale la pena mencionar que este  se refiere a que la experiencia recabada en econofísica relaciona una gran cantidad de datos de varias actividades que realiza la economía. Existen ciertas características universales intrínsecas (por no llamarles leyes) a los diferentes sistemas complejos que se traten, independientemente de las consecuencias del intercambio de bienes, o de la región en el planeta que uno se encuentre, o incluso dentro de una razonable brecha de observación hay independencia del tiempo también (Slanina, F. 2014). 


Ejemplos de lo anteriormente descrito puede ser la cantidad de votos que tiene un candidato a la presidencia de un país, o la cantidad de gazapos que tiene una hembra conejo. Hay modelos que tienen por objetivo es explicar y predecir el comportamiento de fenómenos, en este caso, económicos, cuya tasa de acierto al ser elevada les permite ser relacionados con los llamados \textit{hechos estilizados}, lo cual a su vez incrementa su grado de confianza.    



{
\noindent
\subsection{2.6. Mercado} 
}

Un mercado es un mecanismo a través del cual compradores y vendedores interactúan para determinar precios e intercambiar bienes y servicios (Samuelson \& Nordhaus, 2005).
\newpage


{
\noindent
\subsection{2.7. Mercado financiero} 
}

Sistemas organizados para la compra-venta de acciones, divisas, opciones, bonos, derivados, etc. (Hernández Montoya, 2018).


Mercados cuyos productos o servicios son instrumentos financieros tales como acciones y bonos(Samuelson \& Nordhaus, 2005). 


{
\noindent
\subsection{1.6. Mercado financiero en econofísica} 
}

Sistemas conformados por un número muy grande de agentes que interactúan unos con otros, mismos que reaccionan a información tanto interna como externa a su entorno para determinar el mejor precio según sea el caso.


{
\noindent
\subsection{2.1.2. Economía en el contexto de econofísica} 
}

Le concierne el comportamiento de las personas así como las decisiones que tienen que tomar bajo debidas restricciones. (Hutzler, Mimkes \& Richmond,2013).



Trata las cuestiones que surgen generalmente como resultado intrínseco de estos mecanismos, es decir, la toma de decisiones de un sistema conformado por agentes que llevan a cabo tratos de inversión, compra o venta, dichas cuestiones son denominadas \textit{endógenas}, mientras que las \textit{exógenas} son todas aquellas que si bien no están inmersas en el mecanismo en cuestión, siguen formando parte del sistema complejo, las cuales inlcuyen desde noticias de alto impacto como lo son revoluciones, el inicio de una guerra, desastres naturales, o especulaciones en los mercados financieros.  


Si bien, este trabajo de tesis no se centra en economía, si resulta de suma importancia saber a qué rama de la economía se está estudiando, por esa razón se define \textit{grosso modo} la \textbf{macroeconomía} y la \textbf{microeconomía} a continuación.


{
\noindent
\subsection{2.8. Microeconomía} 
}

Estudia la naturaleza y el comportamiento dinámico de los elementos individuales, como consumidores, dueños de compañías, consejos de administración, quienes interactúan para formar, por ejemplo, precios (Hutzler, Mimkes \& Richmond,2013).


Análisis que explica el comportamiento de elementos individuales de una economía, tales como la determinación del precio de un solo producto o el comportamiento de un solo consumidor o empresa (Samuelson \& Nordhaus, 2005).


{
\noindent
\subsection{2.9. Macroeconomía} 
}

Considera el comportamiento agregado de consumidores, dueños de compañías, consejos de administración, para describir cantidades como el ingreso doméstico bruto, y el nivel de empleo  (Hutzler, Mimkes \& Richmond,2013).


Análisis que trata el comportamiento de la economía en su totalidad con respecto al producto, el ingreso, el nivel de precios, el comercio internacional, el desempleo, y otras variable económicas agregadas. (Samuelson \& Nordhaus, 2005).


{
\noindent
\subsection{1.7. Circuitos económicos} 
}

No abundaremos mucho en este tema ya que escapa del objetivo principal de este trabajo de tesis, sin embargo es importante concebir a los bienes, derivados o acciones, y más concretamente al \textit{dinero} como la \textbf{energía} de un sistema complejo en un contexto meramente financiero, a este tipo de estructuras se les conoce como \textit{sistemas económicos}\footnote{Véase Econophysics \& physical economics,Hutzler, Mimkes \& Richmond,2013 }. 


Un \textbf{circuito económico} es un sistema que puede ser modelado de diferentes maneras, el cual está compuesto por ciclos que pueden ser tan largos como sea permitida la entrada de energía (Hutzler, Mimkes \& Richmond,2013), tomando en cuenta que el balance de energía al final de cada ciclo nunca debe ser negativo. En este trabajo de tesis, se trata con \textit{circuitos monetarios} cuya \textbf{energía} es el \textit{precio} de cierre diario de los mercados.


Cabe destacar que conceptos como el de \textbf{esperanza} o  \textbf{media} no han de sorprender, ya que la esperanza $\boldmath{E}$(x) es un concepto mejor conocido en Física por el nombre de \textit{valor esperado}, de manera análoga y bajo este mismo contexto la media poblacional es la mejor conocida como $\mu$. En aras de no escasear la información que solidifica este trabajo de tesis, se definen a continuación.

{
\noindent
\subsection{1.8. Media de una distribución} 
}

La media $\mu$ de una distribución de probabilidad \textit{P}(\textit{x})para un conjunto finito de valores numéricos \textbf{\textit{x}}, también llamada \textit{esperanza}, \textit{valor esperado} o coloquialmente, \textit{media}, es el promedio de los valores \textbf{\textit{x}} multiplicados por las probabilidades de cada uno: 

\begin{center}
$\mu = \underset{all \hspace{0.15cm} \textbf{\textit{x}}}{\overset{}{\sum}} \textbf{\textit{x}} \hspace{0.11cm} P (x) $
\end{center}
(Probability, Jim Pitman 1993, pg. 162, 163)
\newpage
Si bien el concepto anterior puede entenderse perfectamente para obtener la media de una distribución de valores discretos, también existe la definicion de mediana para valores continuos. A partir de la distribución $\textit{X}$, sea pues $\textit{f(x)}$ una función de densidad para una variable, el valor esperado $\textit{E(X)}$ a menudo denotado solo como $\mu$ o $ \mu_x$ es:

\begin{center}
$\textit{E(X)} =$ $\int_{-\infty}^{\infty} x f(x) dx $
\end{center}
(Time Series Analysis, Cryer Jonathan, Chan Kung-Sik, 2008)


{
\noindent
\subsection{1.9. Mediana} 
}

Sea \textit{m} un número que forma parte de la distribución de \textit{X}, y sea $\textit{x}$ un valor númerico que forma parte de la distribución, la \textit{mediana} es un número \textit{m} tal que tanto la probabilidad de obtener un número $\textit{x}$ menor o igual que $\textit{m}$ y la probabilidad de obtener un número $\textit{x}$ mayor o igual que $\textit{m}$ es igual a $1/2$. (Probability, Jim Pitman 1993, pg. 165)


{
\noindent
\subsection{1.10. Moda} 
}

Sea $\textit{X}$ una distribución, entonces la \textit{moda} es el valor mas probable de $\textit{X}$, y puede haber más de uno valor. (Probability, Jim Pitman 1993, pg. 165)


{
\noindent
\subsection{1.11. Variable aleatoria} 
}

Una variable aleatoria \textit{y} es una función real evaluada definida en el espacio muestra $\mathit{\Omega}$ tal que para cada número real $\mathit{c}$, $A_{c}$ =  \{ $\omega$ $\in$ $\Omega$ $\mid$ \textit{y}($\omega$) $\leq$ $\mathit{c}$ \} $\in$  $\mathcal{F}$ , donde $\mathcal{F}$ es una suma álgebraica de eventos o subconjuntos de $\mathit{\Omega}$. En otras palabras, $A_{c}$ es un evento para el cual la probabilidad es definida en términos de $P_{r}$ que es una medida de probabilidad definida en $\mathcal{F}$. Por completez, la función $\mathit{F}$ : $\mathbb{R}$ $ \rightarrow$ [0,1], definida por $\mathit{F}$($\mathit{c}$) = $P_{r}$($A_{c}$) es la función distribución de  \textit{y}. (New Introduction to Multiple Time Series ,L\"{u}tkepohl, H. 2005, pg. 2)


{
\noindent
\subsection{1.12. Varianza} 
}

Sea \textit{X} una variable aleatoria. La varianza de \textit{X} denotada como $\textit{Var(X)}$  es el cociente de la división entre la raíz cuadrada de la diferencia entre el valor de la variable aleatoria \textit{X} y la media $\mu$ del valor \textit{X} al cuadrado, dividida entre el número de observables. Esto es:

\begin{center}
\hspace{3cm}$\mu = \mu_x = \textit{E(X)}$

$E[(X - \mu)^2]$
\end{center}
(Probability, Jim Pitman 1993, pg. 185)
\newpage

{
\noindent
\subsection{1.13. Distribución normal} 
}

Hasta ahora se ha tratado el concepto de distribución de manera intuitiva y  se ha etiquetado como $\textit{X}$, si bien, puede trabajarse solo con uno de los elementos que le componen, es decir, un elemento $\textit{x}$ o $\textit{x}_i$, así como también puede ser tratada como un todo, o sea, cuando $\textit{X = x}$, en cualquier caso, en el presente trabajo de tesis se trabaja con una \textit{distribución normal}.

La \textbf{distribución normal} es una curva normal que se representa por encima del eje x, el área bajo dicha curva normal se rige por los parámetros de la media $\mu$ y desviación estándar $\sigma$.

\begin{center}
$y = \frac{1}{\sqrt{2\pi}\sigma} e^{\frac{-z^2}{2}}$
\end{center} 

Donde $z = \frac{x - \mu}{\sigma}$ mide el número de desviaciones estándar desde la media $\mu$ hasta el número \textit{x}. Esta distribución posee media cero y desviación estándar uno.( Probability, Jim Pitman 1993, pg. 94)


{
\noindent
\subsection{1.14. Estandarización} 
} 

Proceso aplicado a cada uno de los elementos que pertenecen a la distribución cuyo resultado es la variable aleatoria $z = \frac{x - \mu}{\sigma}$ y como consecuencia se obtiene media cero y desviación estándar uno. (Time Series Analysis, Cryer Jonathan, Chan Kung-Sik, 2008)


{
\noindent
\subsection{1.12. Proceso estacionario} 
}

Se manifiesta cuando las configuraciones de un sistema permanecen inalteradas en el tiempo, y en \textit{Estadística} el concepto es análogo, ya que lo define como \guillemotleft aquel en que las leyes de la probabilidad que gobiernan el comportamiento del proceso no cambian en el tiempo \guillemotright (Chan \& Crier, 2008).
\newpage


{
\noindent
\subsection{1.9. Ruido blanco} 
}
%Otro concepto que vale la pena mencionar es, si bien, tan sutil su presencia en este trabajo de tesis como lo es en la mayoría de los fenómenos observacionales en las ciencias naturales.

%
%Proceso estacionario que es definido como una secuencia de independientes, e  idénticas variables aleatorias distribuídas \{$\textit{y}_{t}$\}. Esto se deja para el capitulo dos. Esto viene en el libro %file:///F:/Springer/2008_Book_TimeSeriesAnalysis.pdf

{
\noindent
\subsection{1.13. Proceso estocástico} 
}

Se define por familias de variables aleatorias  \textit{X}($\mathit{t}; \mathit{\omega})\} \mid \mathit{t} \in \mathbb{T}$, para un conjunto indicado $\mathbb{T}$ (Hassler, U. 2016) :

\begin{center}
\hspace{1.7cm}\textit{X} : $\mathbb{T} \times \mathit{\Omega} \rightarrow \mathbb{R} $ 
($\mathit{t}$ ; $\mathit{\omega}$) $\mapsto$ \textit{X}($\mathit{t}$; $\omega$)
\end{center}


Cabe mencionar que $ \mathit{t} \in \mathbb{T}$ en el contexto de este trabajo de tesis debe ser interpretado como el \guillemotleft tiempo\guillemotright, de tal manera que cuando uno ocupa específicamente un punto o dato en el tiempo $\mathit{t}_{o}$ el proceso estocástico devuelve simplemente una variable aleatoria, 

\begin{center}
\hspace{1.77cm}\textit{X} : $ \mathit{\Omega} \rightarrow \mathbb{R} $ 
$\mathit{\omega}$ $\mapsto$ \textit{X}($\mathit{t_{o}}$; $\omega$)
\end{center}

Es decir que se puede referir a un resultado $\mathit{\omega_{o}}$ en un conjunto, o una \guillemotleft trayectoria \guillemotright realizada por el proceso(Hassler, U. 2016). 


Por su parte, un proceso estocástico \textbf{estacionario} $\mathit{x}$($\mathit{t}$) es aquel proceso estocástico cuya función de densidad de probabilidad (FDP) $P[\mathit{x}(\mathit{t})]$ es invariante bajo un cambio de tiempo (Mantegna R. \& Stanley H., 2004).


En síntesis, es también un objeto \textit{complejo}. Para caracterizarlo matemáticamente, se dispone de vectores aleatorios de longitud finita \textit{n} con puntos arbitrarios en el tiempo $\mathit{t_{1} < ... < \mathit{t_{n}}}$, así, matemáticamente se considera\footnote{(Hassler, U. 2016). }: 
\begin{center}
\textit{X}($\mathit{t_{i}}$) $\equiv$ (\textit{X}($\mathit{t_{
1}}$; $\mathit{\omega}$), ..., \textit{X}($\mathit{t_{
n}}$; $\mathit{\omega}$))' \hspace{0.5cm},\hspace{0.5cm} $\mathit{t_{1}} < \cdots < \mathit{t_{n}}$ 
\end{center}
\newpage
{
\noindent
\subsection{1.14. Serie de tiempo} 
}

Conjunto de variables medidas secuencialmente en el tiempo (Cowpertwait P. \& Metcalfe A, 2009), registradas mediante intervalos de separación entre medición y medición. Su esencia es de un proceso estocástico ya que es fácil reconocer que las observaciones realizadas son impredecibles en el tiempo.


Las series de tiempo financieras conllevan una gran cantidad de información que no es redundante. La cantidad de información es tan grande que es difícil extraer un subconjunto de información económica asociada a algún aspecto (Mantegna R. \& Stanley H., 2004).

Para entender o modelar mecanismos que puedan manifestar comportamientos estocásticos para predecir futuros valores con base en el historial que manifiesta, o con otras series de tiempo cuyos factores son relacionados (Chan \& D. Cryer, 2008).  


En otras palabras, es una secuencia de observaciones a una cantidad o cantidades. (Ruppert D. \& Matteson D, 2015). 


Al utilizar series de tiempo el objetivo es obtener lo que se conoce como  \guillemotleft \textit{ticks} \guillemotright \hspace{0.12cm} que en un sentido muy general representa a cualesquiera variables cotizadas de cualquier origen de cualquier instrumento financiero (Dacoragna M.M., Et Al, 2001), vale la pena mencionar que gracias al poder computacional es posible registrar cada evento (transacción o trato) minuto a minuto, de aquí que en el lenguaje coloquial se utiliza la expresión \guillemotleft tick a tick \guillemotright o el también conocido en inglés\textit{data-tick}  (Hutzler, Mimkes \& Richmond,2013). La secuencia temporal de estos \textit{ticks} es \textit{inhomogénea} en caso de que los intervalos varíen en tamaño, naturalmente si la serie de tiempo cuyos \textit{ticks} registrados son de manera que sus intervalos permancen constantes, se le conoce como \textit{homogénea}. 




{
\noindent
\subsection{2.10. Dos tipos de agentes} 
}


Los mercados que atañen a este trabajo de tesis son conformados por dos tipos de agentes o inversores, los inversionistas \textit{fundamentales}, y los inversionistas \textit{técnicos} (Hutlzer, Mimkes, Richmond, Econophysics \& Physical economics, 2013), en obsequio a la brevedad, se define únicamente a los del segundo grupo, mismos que con la finalidad de no alejarse de la naturaleza de un \textit{modelo de agentes}, se les presenta como \textit{agentes técnicos}.

\newpage
\vspace{0.5cm}
{
\noindent
\subsection{1.16. Tendencias o trends}
}


Cambio en las series de tiempo que puede no ser periódico (Introductory Time Series with R, Cowpertwait, Paul S.P., Metcalfe, Andrew V, 2009, pg. 5)

{
\noindent
\subsection{1.17. Medias móviles} 
}


Sirven para indicar tendencias o \textit{"trends"} en corto o mediano plazo dependiendo de la ventana de tiempo elegida. Mejor conocidad como MAV, por su significado en inglés, \textit{"Moving Average"} o simplemente \textit{"MAV"}.

Asumiendo una longitud de $n$ días para la ventana de tiempo, el periodo en cuestión se etiqueta como $t$ y el precio de cierre que registra cada uno de los días es $C_t$, la media móvil simple se define como sigue: 

\begin{center}
$MAV = \biggl( \frac{1}{n} \biggr)(C_{t} + C_{t-2} + \cdots C_{t-n+1})$
\end{center}

Aunque suelen utilizarse con los precios de cierre, pueden ser utilizados con los precios en lo que abre un mercado, los precios más altos o los más bajos. (Mohammadali Mehralizadeh, 2017 pg.55)

{
\noindent
\subsection{2.10.1. Agentes técnicos} 
}


En Economía son mejor conocidos como \textit{"traders" técnicos} o \textit{inversionistas técnicos}. Son agentes que ignoran por completo los fundamentos en que se basan los \textit{"traders" fundamentalistas}, cuyo criterio de decisión se basa en la apreciación numérica, matemática y estadística de los datos de un mercado. \vspace{0.5cm}

El análisis que realizan estos agentes es, como su nombre lo dice, técnico, por lo que el precio, la duración de ese precio, el volumen de interacciones realizadas en un intervalo de tiempo, y las tendencias que se pueden observar en la serie de tiempo del mercado son solo unos de los muchos elementos en los que un agente técnico se basa para tomar decisiones, e incluso, predicciones.


{
\noindent
\subsection{1.16. Modelos fundamentales en econofísica} 
}

Debido a la realción que tiene la \textit{Econofísica} con la \textit{mecánica estadística}, es momento de sentar bases respecto a modelos como el de Boltzmann y Gibss, dado que será de suma utilidad relacionar estas bases con el tema medular de este trabajo de tesis.

\vspace{0.5cm}


{
\noindent
\subsection{1.16.2. Ley de Boltzmann-Gibbs} 
}

La \textit{ley de la distribución de la probabilidad de la energía} en la \textit{mecánica estadística}:

\begin{center}
$\mathit{P(\varepsilon)} = $ C $e^{-\varepsilon/\mathit{T}}$ 
\end{center}
Donde:

$\varepsilon$ es la energía

$C$ es una constante de normalización

$\mathit{T}$ es la temperatura (Wannier,1966) (Cottrell, Michelson, Wright, Classical Econophysics, pg.149)

Dada la similitud de un sistema descrito por la Ley de Boltzmann-Gibbs y las condiciones de un sistema económico cerrado, se puede esta ley a una descripción de la \textit{distribución de la probabilidad del dinero}.
(Cottrell, Michelson, Wright, Classical Econophysics, pg.149)
\\

{
\noindent
\subsection{1.16.3. Distribución de la probabilidad del dinero (en equilibrio)} 
}


De acuerdo con Cottrel, Michelson y Wright (Cottrell, Michelson, Wright, Classical Econophysics, pg.149) la distribución de la probabilidad del dinero sigue la forma de la \textit{Ley de Boltzmann-Gibbs}


\begin{center}
$\mathit{P(d)} = $ C $e^{-d/\mathit{T}}$ 
\end{center}
Donde:

$d$ es el dinero

$C$ es una constante de normalización

$\mathit{T}$ es el promedio de la cantidad de dinero por agente en el sistema económico.

Con base en el artículo \textit{Price variations in a stock market with many agents} por Shubik, Pakzuski y Bak en 1997, la ley de la conservación del dinero establece que el dinero no puede ser manufacturado por agentes del sistema económico pero si puede ser transferido entre ellos. Esta descripción es análoga a la que un sistema conservativo presenta en Física, por ejemplo, cuando en un sistema cerrado sin intercambio de energía con el exterior, los átomos colisionan entre sí. 
\newpage



%Ahora que se ha comprendido que el intercambio de dinero $\Delta d$ entre %los agentes es la interacción elemental para la existencia de la %\textit{Economía}, es momento de entender el funcionamiento de la %\textit{Economía neoclásica} para poder utilizar el concepto de %\textit{caminata aleatoria} mismo que se utiliza para sentar las bases del %concepto de \textit{mercado eficiente}.

{
\noindent
\subsection{2.11.1. La mano invisible} 
}

Idea propuesta por Adam Smith en  1776, guiada por los ideales del \textit{laissez-faire}, que significa "dejar hacer", la cual se refiere a que cuando cada participante busca solo su propio beneficio, a menudo promueve el beneficio de la sociedad de manera más eficaz que si realmente pretendiera promoverlo, como si una mano invisible benevolente estuviera dirigiendo todo el proceso. (Economía, Samuelson Nordhauss, pg.26)




{
\noindent
\subsection{2.11.2. Teoría de juegos} 
}

Estudio que puede ser descrito con el uso de las Matemáticas para comprender la competencia y cooperación entre varias partes involucradas . El rango de aplicación va desde estrategias de guerra hasta el entendimiento de la competición económica, desde problemas económicos o sociales hasta el comportamiento de los animales en situaciones de competencia(Peters, Hans. 2015, Springer, Game Theory).

Análisis de situaciones que comprende a dos o más tomadores de decisiones con intereses al menos parcialmente en conflicto. Se puede aplicar a la interaccion de dos o más agentes en situaciones de negociación como las huelgas, o conflictos como los juegos y la guerra(Samuelson, Nordhauss, Economía).


{
\noindent
\subsection{2.11.3. Dilema del prisionero} 
}

Existen diferentes problemas que estudia la disciplina de la \textit{teoria de juegos}, el que se presenta a continuación es probablemente el más conocido y el que mejor puede extrapolarse a este trabajo de tesis. 

Sean dos prisioneros P1 y P2  que han cometido juntos un delito. El fiscal entrevista por separado a cada uno de ellos y les dice : \textit{"Tengo suficientes pruebas sobre los dos para mandarlos un año a la cárcel. Pero haré un trato contigo: si solo confiesas tu, se te condenarpa a tres meses de cárcel, mientras que tu socio será condenado a 10 años. Si confiesan los dos, ambos serán condenados a cinco años"}. 

Lo importante en este caso es que cuando ambos prisioneros actúan interesadamente y confiesan, ambos serán condenados a mayores penas de cárcel. Solo son condenados a menos años de carcel cuando ambos actúan de manera colusiva y altruísta. La situación en cuestión concluye que el equilibrio no cooperativo es ineficiente. 
\newpage

{
\noindent
\subsection{2.11.4. Frecuencia} 
}

Si bien, en \textit{Física} la frecuencia es medida en \textbf{Hz} debido a la ecuación:
\begin{center}
$f=\frac{1}{\textit{T}}$
\end{center}

Donde: 

$\textit{T}  $ es el periodo y se mide en segundos.

En el contexto de este trabajo de tesis se entiende la \textit{frecuencia} como una proporción cuantificable de registros de observables, datos, o precios en el tiempo (Jim Pitman, Probability, Springer, 1993).

{
\noindent
\subsection{2.11.5. Fluctuación} 
}


Observable registrada en el tiempo tan grande o tan pequeña conforme con el total de observables y la frecuencia de cada una. 

{
\noindent
\subsection{2.11.6. Volatilidad} 
}

En \textit{Economia} a menudo es calculada como la desviación estándar del cambio en el precio en una apropiada ventana de tiempo. (Mantegna, Econophysics).

Aunque se conocen la \textit{volatilidad completa o histórica}, la \textit{volatilidad implícita}, la  \textit{volatilidad instantánea o actual}, la \textit{volatilidad de frontera} y la \textit{volatilidad progresiva}, siendo esta última una de las utilizadas en predicciones. Esta tesis se centra específicamente en la primera.

Es una medida de aleatoriedad de un conjunto de registros para dos puntos cualesquiera en el tiempo (Wilmott, FAQ, Quantitative Finance). 

Asumiendo que existe una ventana de tiempo de interés  $t_{i}$ La siguiente ecuación describe la volatilidad en cuestión:

\begin{center}

$\textit{v}(t_{i}) = \textit{v}(\Delta t, n, p; t_{i}) = \Bigl[ \frac{1}{n} \sum_{j = 1}^{n} |r (\Delta t ; t_{i-n+j}|^{\textit{p}} \Bigr]^\frac{1}{\textit{p}}$

\end{center}

Donde $\textit{r}$ son los retornos que se definen en la sección \textbf{retornos}, $\textit{n}$ es el número de la cantidad de observables utilizada para calcular los retornos, y $\textit{p}$ suele tomar el valor de 2 ya que $\textit{v}^{2}$ es la varianza de los retornos.  (An introduction to high-frequency finance, Dacoragna, Gencay, Muller, Olsen, Pictet 2001)

Antes de definir el siguiente concepto vale la pena mencionar que en este contexto la palabra \textit{eficiencia} se refiere a que la información se interpreta con rapidez, y no se refiere a los recursos que generen una producción máxima \citep[][]{samuelson2009economics}.

\subsection{Mercado eficiente}
Pesaran definition

At the core of the EMH lies the following three basic premises:
23
1. Investor rationality: It is assumed that investors are rational, in the sense that
they correctly update their beliefs when new information is available.
2. Arbitrage: Individual investment decisions satisfy the arbitrage condition, and
trade decisions are made guided by the calculus of the subjective expected
utility theory a la Savage.
3. Collective rationality: Di§erences in beliefs across investors cancel out in the
market.



In his 1970 review, Fama distinguishes between three di§erent forms of the EMH:
1. The weak form asserts that all price information is fully reáected in asset prices,
in the sense that current price changes can not be predicted from past prices.
This weak form was also introduced in an unpublished paper by Roberts (1967).
2. The semi-strong form that requires asset price changes to fully reáect all publicly available information and not only past prices.
3. The strong form that postulates that prices fully reáect information even if
some investor or group of investors have monopolistic access to some information.


En analogía con lo que exponen Samuelson y Nordhauss, si una compañía minera hiciera el hallazgo de una mina de diamentes a las 10:00 horas, las acciones de dicha compañía se elevarían de precio ya que los inversionistas de dicha compañía invertirían de inmediato.

 \textit{En forma breve, en todo momento, los mercados ya incluyeron toda la información correspindiente disponible en los precios de las acciones}  \citep[][]{samuelson2009economics}.
 
 


La hipótesis del mercado eficiente sostiene que los movimientos de los precios accionarios deben ser muy erráticos, como un camino aleatorio, cuando se los grafica durante un cierto periodo. 

En un mercado eficiente, todas las cosas predecibles ya se han incorporado al precio. Es la llegada de nueva información la que afecta los precios
de las acciones o los bienes, y dichos precios son los que parecieran estar comportándose aleatoriamente \citep[][]{samuelson2009economics}.
